\documentclass[11pt]{article}
%Gummi|065|=)
\title{\textbf{Contest Analysis - CF Round \#480}}
\author{Anurag Jain}
\date{}
\begin{document}

\maketitle

\section{Brief}

This was a relatively difficult contest.

\begin{tabular}{|cc|}
\hline
	Questions Solved &  Rank\\
	3 & 1000\\
	2 & 2000\\
	1 & 3500\\
\hline
\end{tabular}

The B Question was tricky and the solution was not obvious.
An ideal strategy would have been A $\rightarrow$ C $\rightarrow$ B.

\section{980 B - Marlin}
It was a tricky one, and the solution required a lot of thinking.
\underline{\textbf{Observations :}}
    \begin{itemize}
      \item No. of Paths from the city to pond will be same as No. of Paths from pond to city.
      \item This means that there exists an axis of symmetry in vertical direction in the middle.
      \item Along with this, there is also an axis of symmetry in horizontal direction but it is not of much use.
    \end{itemize}
  
Now when 
k
 is even, we can do the following: start from (2,2), put a hotel in that cell and another hotel in (2,n-1), then a hotel in (2,3) and one in (2, n-2), and so on until either the second row is full (expect for the middle column) or the number of hotels is 
k
, if you still need more hotels then just do the same for the third row.

This works because going from 
(
1
,
1
)
 to 
(
4
,
n
)
 is identical to going from 
(
1
,
n
)
 to 
(
4
,
1
)
 since the constructed grid is symmetric.

Now if 
k
=
2
∗
(
n
−
2
)
 then just fill the the middle column (second and third row), and if 
k
 is odd then just add a hotel in middle column.

\section{980 C - Posterized}
If you are wondering where your old default text is; it has been stored as a template. The template menu can be used to access and restore it. 

\end{document}
